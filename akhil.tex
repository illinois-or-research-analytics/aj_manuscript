\documentclass{article}
\begin{document}

The kmp-clustering pipeline takes as input a network N and values for parameters $k$ and $p$. It then has 4 steps.
\begin{itemize}
\item Step 1: IKC(N,k): producing a set of disjoint clusters, each of which is k-valid.
\item Step 2 (optional). Divide into smaller clusters
\item Step 3 (optional): augment for periphery
\item Step 4 (not optional): parse again, making sure to update for kmp-validity
\end{itemize}

What we propose to do is add a Step 5 that will achieve overlapping clusters. More generally, the Step 5 will be able to be used with
any input clustering (even if not disjoint) of a network N. And if given values for k and p, will maintain kmp-validity, if desired.
Furthermore, we have a variant of this Step 5 that will maintain MCD (minimum core degree).


\section{Proposed greedy algorithm}

Suppose we have as input a network $N$, values for $k$ and $p$, and a clustering $\mathcal{C}$, and we want to now allow for nodes to be members in more than one cluster.  Thus we want to enhance $\mathcal{C}$ to create a new clustering, but using $\mathcal{C}$ as a starting point.
Here is a general technique:

\begin{itemize}
\item Sort the nodes of $N$ according to some criterion (which is referred to as the candidate selection in Akhil's writeup)
\item Process the nodes in order of this criterion, from best to worst, until a stopping condition applies (could be the size of the resulting 
clustering, or amount of time that has passed)
\begin{itemize}
\item Given node $v$, add $v$ to any cluster in $C \in \mathcal{C}$ where $v$ has at least $k$ neighbors (alternatively, 
$v$ has at least $MCD(C)$ neighbors) among the core elements of $C$. Potentially, also require that adding $v$ to  $C$ does not
undo positive modularity.
\item Consider a variant where we add nodes to clusters based on how many core members cite the nodes, rather than just number of neighbors.
\end{itemize}
\end{itemize}

Note: for the first test of the new algorithm, let's not change the center as we augment in the greedy algorithm. 


\section{Proposed study}
We propose to begin with the output of kmp-clustering, run in two different ways, and follow by this greedy algorithm.

\begin{itemize}
\item Try $k=10$ and $k=20$, and don't even bother with periphery construction (drop Steps 3 and 4).
\item Run two variants: one where we optimize MCD entirely (and so omit Step 2) and require maintenance of MCD, and the other where we just maintain kmp-validity, and so allow for Step 2.

\end{itemize}

Two stopping conditions to consider.
\begin{itemize}
\item For a specified node that is being processed, how many of the clusters to look at? We look at them all.
\item For which nodes can be put in more than one cluster?  Try the top 1\% in terms of total degree. 
\end{itemize}




ALso, what do we want to report?
\begin{itemize}
\item Total edge coverage
\item Node coverage
\item Number of clusters
\item Number of singletons
\item What else?
\end{itemize}
\end{document}
~                                                                               
~                                                                               
~                                                                               
~                                                                               
~                                                                               
~                                                                               
~                                                                               
~                                                                               
~                                                                               
~                                                                               
~                                                                               
~                                                                               
~                                                                               
~                                                                               
~                                                                               
~                                                                               
~                                                                               
~                                                                               
~                                                                               
~                                                                               
-- INSERT --
