\documentclass[12pt, oneside]{article}   	% use "amsart" instead of "article" for AMSLaTeX format
\usepackage{textcomp}
\usepackage{geometry}                		% See geometry.pdf to learn the layout options. There are lots.
\geometry{letterpaper}                   		% ... or a4paper or a5paper or ... 
%\geometry{landscape}                		% Activate for rotated page geometry
%\usepackage[parfill]{parskip}    		% Activate to begin paragraphs with an empty line rather than an indent
\usepackage{graphicx}				% Use pdf, png, jpg, or eps§ with pdflatex; use eps in DVI mode
\usepackage{caption}
\usepackage{subcaption}								% TeX will automatically convert eps --> pdf in pdflatex		
\usepackage{hyperref}
\usepackage{amssymb}
\usepackage{listings}
\usepackage{amsthm}
\newtheorem{theorem}{Theorem}
\newtheorem{definition}{Definition}
\usepackage{natbib}
\usepackage{xcolor}
\usepackage{hyperref}
\usepackage{authblk}
\usepackage{amsmath}
\hypersetup{
    colorlinks=false,
    linkcolor=blue,
    filecolor=magenta,      
    urlcolor=blue,
    pdfpagemode=FullScreen,
}
\usepackage{float}
\usepackage{rotating}
\usepackage{adjustbox}
\usepackage[font=small,labelfont=bf]{caption}

\usepackage{authblk}
\title{Supplementary Materials for AOC: Assembling Overlapping Clusters}
\author[1]{Akhil Jakatdar\thanks{akhilrj2@illinois.edu}}
\author[1]{Baqiao Liu\thanks{baqiaol2@illinois.edu}}
\author[1]{Tandy Warnow\thanks{warnow@illinois.edu}}
\author[1,2]{George Chacko\thanks{chackoge@illinois.edu}}
\affil[1]{Department of Computer Science, University of Illinois Urbana-Champaign, Urbana, IL 61801}
\affil[2]{Office of Research, Grainger College of Engineering, University of Illinois Urbana-Champaign, Urbana, IL 61801}

% \setlength{\parindent}{0pt}
%SetFonts
\definecolor{codegreen}{rgb}{0,0.6,0}
\definecolor{codegray}{rgb}{0.5,0.5,0.5}
\definecolor{codepurple}{rgb}{0.58,0,0.82}
\definecolor{backcolour}{rgb}{0.95,0.95,0.92}
\lstdefinestyle{mystyle}{
    backgroundcolor=\color{backcolour},   
    commentstyle=\color{codegreen},
    keywordstyle=\color{magenta},
    numberstyle=\tiny\color{codegray},
    stringstyle=\color{codepurple},
    basicstyle=\ttfamily\footnotesize,
    breakatwhitespace=false,         
    breaklines=true,                 
    captionpos=b,                    
    keepspaces=true,                 
    numbers=left,                    
    numbersep=5pt,                  
    showspaces=false,                
    showstringspaces=false,
    showtabs=false,                  
    tabsize=2
}

\lstset{style=mystyle}

\begin{document}
\maketitle

\clearpage

\section{Commands for Assembling Overlapping Cluster} 

After downloading \href{https://github.com/illinois-or-research-analytics/aocv2_plus/tree/main/v2_revisions}{our software} locally, the following command can be run to generate the set of overlapping clusters:

 \lstset{
  breaklines=true,
  literate = {-}{-}1
}
\begin{lstlisting}[language=bash]
python3 aoc.py
--network-file [path_to_tsv_network_edgelist]
--clustering [existing_cluster_by_ikc]
--output-path [path_to_output_overlapping_clustering]
--min-k-core [10, 20, 30, 40, 50]
--inclusion-criterion [k|mcd]
--candidate-file [path_to_candidate_file]
\end{lstlisting}

Shown below are some explanations for the parameters (for more information, see our README for the software):

\begin{itemize}
\item \textbf{network-file} - the path to the background graph, in edge-list format
\item \textbf{clustering} - the path to an input disjoint clustering, each line contains an entry of format \path{cluster_id[\space]node_id}
\item \textbf{output-path} - the path for the output clustering
\item \textbf{min-k-core} - an integer defining the threshold for $k$ by which to add candidate nodes if using ``AOC\_k''. Not required if using ``AOC\_m''
\item \textbf{inclusion-criterion} - use \texttt{k} (``AOC\_k'') or \texttt{mcd} (``AOC\_m'') as the inclusion criteria for adding a candidate node to a cluster
\item \textbf{candidate-file} - the path to the candidate nodes. Not required and defaults to all nodes appearing in the input clustering
\end{itemize}

For example, given a network and clustering files produced by IKC\_k10, and assuming
that we want the candidates to be all nodes that appear in one of the input non-singleton clusters, AOC\_k can be run as follows:
\begin{lstlisting}[language=bash]
python3 aoc.py --network-file [path_to_tsv_network_edgelist] --clustering [existing_cluster_by_ikc]
    --output-path [output_path]
    --min-k-core 10
    --inclusion-criterion k
\end{lstlisting}

\section{Line Graph Proof} 

We here show in details our derivation for the size of the corresponding line graph of the Curated Exosome Network (CEN), which has 
13,989,436 vertices and 92,051,051 edges.

\subsection{Derivation of Line Graph Nodes and Edges}

In transforming any undirected graph $G(V, E)$ into its corresponding line graph $L(G(V', E'))$, the number of vertices $|V'| = |E|$ as all edges in the original graph are transformed into vertices in its corresponding line graph.

We can also see that every edge $e' \in E'$ can be described as a set of 2 edges $\{e_1, e_2\}$ where $e_1,  e_2 \in E$ where both edges are adjacent to some common vertex $v \in V$.  
Thus in order to find the total number of edges $e' \in E'$ for some fixed vertex $v \in V$, we formulate the number of edges as $d_v \choose 2$ where $d_v$ represents the degree of $v$. Thus summing over all vertices in $V$, we get the following:
\begin{equation}
    \begin{aligned}
    |E'| &= \sum^{V}_{v} {d_v \choose 2}\\
    &= \sum^{V}_{v} \frac{d_v(d_v - 1)}{2}\\
    &= \sum^{V}_{v} \frac{{d_v}^2}{2} - \sum^{V}_{v} \frac{d_v}{2}\\
    &= \frac{1}{2}\sum^{V}_{v} {d_v}^2 - \sum^{V}_{v} \frac{d_v}{2}
    \end{aligned}
\end{equation}
The expression $\sum^{V}_{v} \frac{d_v}{2}$ can be simplified to $|E|$ as the sum of the degrees of all vertices in $G$ divided by 2 is equivalent to the total number of edges in $G$. Thus our closed form equation for $|E'|$ is as follows:
\begin{equation}
    \begin{aligned}
    |E'| &= \frac{1}{2}\sum^{V}_{v} {d_v}^2 - |E|
    \end{aligned}
\end{equation}


We can now use these two closed form solutions to identify the number of edges and vertices in the corresponding line graph to the Exosome Citation Network. Given that this network $N(V_n, E_n)$ has $|V_n| = 13989436$ vertices and $|E_n| = 92051051$, we can solve for $|V_n'| = 92051051$ and plug in the degree distribution of the network into our closed form solution for the number of edges to find the following:
\begin{equation}
    \begin{aligned}
    |E_n'| &= \frac{1}{2}\sum^{V_n}_{v} {d_v}^2 - |E_n|\\
    &= \frac{1}{2}(321041864344) - 92051051
    &= 160428881121
    \end{aligned}
\end{equation}
Thus, the corresponding line graph $L(G_n(V_n', E_n'))$ to our Exosome Citation Network will have $|V_n'| = 92,051,051$ and $|E_n'| = 160,428,881,121$. 



\bibliographystyle{apalike}
\bibliography{akhil.bib}
\end{document}



